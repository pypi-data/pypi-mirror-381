
\documentclass[12pt]{article}
%%%%%%%%%%%%%%%%%%%%%%%%%%%%%%%%%%%%%%%%%%%%%%%%%%%%%%%%%%%%%%%%%%%%%%%%%%%%%%%%%%%%%%%%%%%%%%%%%%%%%%%%%%%%%%%%%%%%%%%%%%%%%%%%%%%%%%%%%%%%%%%%%%%%%%%%%%%%%%%%%%%%%%%%%%%%%%%%%%%%%%%%%%%%%%%%%%%%%%%%%%%%%%%%%%%%%%%%%%%%%%%%%%%%%%%%%%%%%%%%%%%%%%%%%%%%
\usepackage{amsfonts}
\usepackage{amssymb}
\usepackage{sw20elba}

%TCIDATA{OutputFilter=LATEX.DLL}
%TCIDATA{Version=5.50.0.2890}
%TCIDATA{<META NAME="SaveForMode" CONTENT="1">}
%TCIDATA{BibliographyScheme=Manual}
%TCIDATA{Created=Friday, July 19, 2013 14:12:36}
%TCIDATA{LastRevised=Saturday, July 20, 2013 08:24:13}
%TCIDATA{<META NAME="GraphicsSave" CONTENT="32">}
%TCIDATA{<META NAME="DocumentShell" CONTENT="Articles\SW\mrvl">}
%TCIDATA{CSTFile=LaTeX article (bright).cst}

\newtheorem{theorem}{Theorem}
\newtheorem{axiom}[theorem]{Axiom}
\newtheorem{claim}[theorem]{Claim}
\newtheorem{conjecture}[theorem]{Conjecture}
\newtheorem{corollary}[theorem]{Corollary}
\newtheorem{definition}[theorem]{Definition}
\newtheorem{example}[theorem]{Example}
\newtheorem{exercise}[theorem]{Exercise}
\newtheorem{lemma}[theorem]{Lemma}
\newtheorem{notation}[theorem]{Notation}
\newtheorem{problem}[theorem]{Problem}
\newtheorem{proposition}[theorem]{Proposition}
\newtheorem{remark}[theorem]{Remark}
\newtheorem{solution}[theorem]{Solution}
\newtheorem{summary}[theorem]{Summary}
\newenvironment{proof}[1][Proof]{\noindent\textbf{#1.} }{{\hfill $\Box$ \\}}
\input{tcilatex}
\addtolength{\textheight}{30pt}

\begin{document}

\title{Algebra 6.150}
\author{Michael Vaughan-Lee}
\date{July 2013}
\maketitle

Algebra 6.150 has presentation 
\[
\langle a,b,c\,|\,ca-baa,\,cb,\,pa,\,pb,\,pc,\,\text{class }3\rangle 
\]%
and so if $L$ is a descendant of 6.150 of order $p^{7}$ then the commutator
structure of $L$ is the same as one of 7.108 -- 7.115 from the list of
nilpotent Lie algbebras over $\mathbb{Z}_{p}$. So we can assume that one of
the following sets of commutator relations holds. 
\begin{eqnarray*}
baaa &=&baab=ca-baa=cb=0, \\
baaa &=&baab=ca-baa-babb=cb=0, \\
baaa &=&baab=babb,\,ca-baa=cb=0, \\
babb &=&-baaa,\,baab=ca-baa=cb=0, \\
babb &=&-\omega baaa,\,baab=ca-baa=cb=0, \\
baaa &=&babb=ca-baa=cb=0, \\
babb &=&baab,\,baaa=ca-baa=cb=0, \\
baaa &=&baab,\,babb=xbaaa,\,ca-baa=cb=0\;(x\neq 0,1).
\end{eqnarray*}%
For each of these cases we obtain a generator $d$ for $L_{4}$ ($d$ equals
one of $baaa$, $baab$, $babb$) and we write 
\[
\left( 
\begin{array}{l}
pa \\ 
pb \\ 
pc%
\end{array}%
\right) =Ad
\]%
where $A$ is a $3\times 1$ matrix over $\mathbb{Z}_{p}$. In each case the
isomorphism classes of algebras are given by the orbits of the matrices $A$
under a given action by a group of automorphisms. I was able to
\textquotedblleft solve\textquotedblright\ the problem in every case,
providing presentations with fewer parameters, and explicit relatively
simple equivalence relations on the parameter sets. However in four of the
cases the equivalence classes were slightly more complex than usual. For
example in one case the equivalence classes for a parameter $y$ were $\{\pm
y,\pm \frac{\omega }{y}\}$. These four cases were 3,4,5,8, and these are
described below. There are \textsc{Magma} programs to compute a
representative sets of matrices $A$ in these four cases in notes6.150.m.

\subsubsection{Case 3}

If $baaa=baab=babb,\,ca-baa=cb=0$ then $L_{4}$ is generated by $baaa$ and
the action on $A$ is 
\[
A\rightarrow \alpha ^{-4}\left( 
\begin{array}{lll}
\alpha  & 0 & \gamma  \\ 
0 & \alpha  & -\gamma  \\ 
0 & 0 & \alpha ^{2}%
\end{array}%
\right) A
\]

and 
\[
A\rightarrow -\alpha ^{-4}\left( 
\begin{array}{lll}
0 & \alpha  & \gamma  \\ 
\alpha  & 0 & -\gamma  \\ 
0 & 0 & \alpha ^{2}%
\end{array}%
\right) A.
\]

\subsubsection{Case 4}

If $babb+baaa=baab=ca-baa=cb=0$ then $L_{4}$ is generated by $baaa$ and the
action on $A$ is 
\[
A\rightarrow \alpha ^{-4}\left( 
\begin{array}{lll}
\pm \alpha  & 0 & \gamma  \\ 
0 & \alpha  & \varepsilon  \\ 
0 & 0 & \alpha ^{2}%
\end{array}%
\right) A
\]

and 
\[
A\rightarrow \alpha ^{-4}\left( 
\begin{array}{lll}
0 & \pm \alpha  & \gamma  \\ 
\alpha  & 0 & \varepsilon  \\ 
0 & 0 & \alpha ^{2}%
\end{array}%
\right) A.
\]

\subsubsection{Case 5}

If $babb+\omega baaa=baab=ca-baa=cb=0$ then $L_{4}$ is generated by $baaa$
and the action on $A$ is 
\[
A\rightarrow \alpha ^{-4}\left( 
\begin{array}{lll}
\pm \alpha  & 0 & \gamma  \\ 
0 & \alpha  & \varepsilon  \\ 
0 & 0 & \alpha ^{2}%
\end{array}%
\right) A
\]

and 
\[
A\rightarrow \omega ^{-2}\alpha ^{-4}\left( 
\begin{array}{lll}
0 & \pm \alpha  & \gamma  \\ 
\omega \alpha  & 0 & \varepsilon  \\ 
0 & 0 & \omega \alpha ^{2}%
\end{array}%
\right) A.
\]

\subsubsection{Case 8}

If $baaa=baab,\,babb=xbaaa,\,ca-baa=cb=0$ where $x\neq 0,1$ then $L_{4}$ is
generated by $baaa$. (If we set $x=0$ we have Case 7, and if we set $x=1$ we
have Case 3.) The action on $A$ is 
\[
A\rightarrow \alpha ^{-4}\left( 
\begin{array}{lll}
\alpha  & 0 & \gamma  \\ 
0 & \alpha  & \varepsilon  \\ 
0 & 0 & \alpha ^{2}%
\end{array}%
\right) A
\]%
and 
\[
A\rightarrow x^{-2}\alpha ^{-4}\left( 
\begin{array}{lll}
0 & \alpha  & \gamma  \\ 
x\alpha  & 0 & \varepsilon  \\ 
0 & 0 & -x\alpha ^{2}%
\end{array}%
\right) A.
\]

\end{document}
