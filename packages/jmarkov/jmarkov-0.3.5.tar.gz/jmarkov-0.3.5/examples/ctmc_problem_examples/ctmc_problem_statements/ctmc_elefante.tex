%CMTC: Elefante en Circo
% \hfill {\footnotesize \textbf{CTMC}} 

Rosie es un elefante hembra y la mayor atracción del circo Benzini Brothers. Después de cada función, hay una sesión en la que los clientes se toman una foto montados en Rosie. Los clientes llegan a tomarse fotos siguiendo un proceso de Poisson con tasa $\lambda = 0.5 \text{ min}^{-1}$. El tiempo que tarda una persona mientras se le ayuda a subir al elefante, se toma la foto y baja del elefante se distribuye exponencial con media $\frac{1}{\mu} = 4$ minutos. Dado que Rosie es muy nerviosa no pueden haber más de 6 clientes en su carpa. Rosie tarda un tiempo exponencial con media $\frac{1}{\theta} = 45$ minutos en agotarse y demora descansando un tiempo exponencial con media $\frac{1}{\omega} = 10$ minutos. Mientras Rosie está descansando, los clientes que están tanto siendo atendidos como esperando permanecen en el sistema.

\begin{enumerate}
\item Modele la situación de Rosie como una cadena de Markov de tiempo continuo. \

    \begin{itemize}
    	\item[] \textbf{Variables de estado}:\\
    	$X(t)$: Número de clientes en el circo en el momento $t$
            \[Y(t) = \begin{cases} 
                R & \text{si Rosie está descansando} \\
                W & \text{si Rosie está trabajando} 
                \end{cases}
            \]
            \[ W(t) = \{ X(t), Y(t) \} \]
            
    		
    	\item[] \textbf{Espacios de estados}:\\
    	$S_X = \{0,1,2,3,4,5,6\}$ \\
            $S_Y = \{R,W\}$ \\
            $S_W = \{S_X \times S_Y\}$ \\

    	\item[] \textbf{Tasas de transiciones}:\

            \begin{align*}
             q_{{i,j} \to {i',j'}} = \left\{ 
                \begin{array}{llll}
                    \lambda    &  i'=i+1, j'=j & i<6 \\ 
                    \mu    &  i'=i-1, j'=j & i>0, j=W \\ 
                    \theta    &  i'=i, j'=R & j=W \\ 
                    \omega    &  i'=i, j'=W & j=R \\
                    0 & \text{d.l.c.}
                \end{array} \right.
            \end{align*}

    
    \end{itemize}



\end{enumerate}
